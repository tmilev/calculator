\documentclass{article}
\usepackage{amsmath, amsthm, amsfonts, amssymb, verbatim, hyperref}
\newtheorem{theorem}{Theorem}[section]
\newtheorem{lemma}[theorem]{Lemma}

\begin{document}

\section{Hensel lifts}
Let $p$ be a prime and suppose we have a polynomial factorization
\[
u = \bar v \cdot \bar w \mod p
\] with \(p\)-prime, where \(\bar v\) and \(\bar w\) are relatively prime monic polynomials in the variable \(x\). Set 
\[
\deg \bar v = m\quad \deg \bar  w = n.
\] 
Our goal is to construct monic polynomials \(v, w\) of the same degrees as \(\bar v,\bar  w\) with 
\[
u = v \cdot w \mod p^k.
\]
Since factorization $\mod p$ is unique, it follows that \(v =  \bar v \mod p \) and \(w = \bar w \mod p\). Our secondary goal is to show that, \(\mod p^{k}\), these polynomials are unique. Proceed by induction. Suppose we have already found \(v, w\) as above and are looking for \(v', w'\) with 
\[
u = v'\cdot w' \mod p^{k+1}.
\]
Over $\mathbb Z$, we have that

\[ 
u = vw + p^k z 
\]
for some polynomial \(z\). Since \(u,v,w\) are monic, the leading monomials of \(u\) and \(vw\) are $x^{m+n}$ and so $\deg z = m + n - 1$. Since  \(v, w\) are unique \(\mod p^{k}\), it follows that \(v', w'\) must be of the form 
\[
v' = v + p^{k} a \quad w' = w+ p^k b
\]
with \(\deg a < \deg v= m\) and \(\deg b < \deg w = n\). 
\begin{equation}\label{eqHenselLift0}
\begin{array}{rcl}
v' w'& =& \left(v+p^k a\right)\left(w + p^k b\right) \\
&=&vw + p^k(aw + bv) + p^{2k} ab\\
&=&u + p^{k}z + p^{k}(aw+bv) + p^{2k}ab\\
&=&u + p^{k}\left( z + aw + bv\right) + p^{2k}ab
\end{array}
\end{equation}
It follows that 
\begin{equation}\label{eqHenselLift1}
z + aw + bv = 0 \mod p
\end{equation} 
Introduce notation for the coefficients of the polynomials in \eqref{eqHenselLift1}:
\begin{equation}
\begin{array}{rcl}
a &=& a_{1} x^{m-1} + \dots + a_m\\
b &=& b_{1} x^{n-1} + \dots + b_m \\
z &=& z_{1}x^{m + n - 1} + \dots +z_{m+n}\\
w &=& x^{m} + w_{1} \cdot x^{m - 1} + \dots + w_{m}\\
v &=& x^{n} + v_{1} \cdot x^{n - 1} + \dots + v_{n}\\
\end{array}
\end{equation}
Then \eqref{eqHenselLift1} is a linear system in the \(m+n\) variables \(a_1, \dots, a_m, b_1, \dots, b_n\) with $\deg z + 1 = m + n $ equations. More precisely, the matrix form of the equation \eqref{eqHenselLift1} is given by the Sylvester matrix:
\begin{equation}\label{eqHanselLift2Sylvester}
\left( \begin{array}{ccccccccccc}
w_0    & 0       & \cdots & 0       & v_0    & 0       & \cdots & 0      \\
w_1    & w_0     & \cdots & 0       & v_1    & v_0     & \cdots & 0      \\
w_2    & w_1     & \ddots & 0       & v_2    & v_1     & \ddots & 0      \\
\vdots &\vdots   & \ddots & w_0     & \vdots &\vdots   & \ddots & v_0    \\
w_m    & w_{m-1} & \cdots & \vdots  & v_n    & v_{n-1} & \cdots & \vdots  \\
0      & w_m     & \ddots & \vdots  & 0      & v_n     & \ddots & \vdots  \\
\vdots & \vdots  & \ddots & w_{m-1} & \vdots & \vdots  & \ddots & v_{n-1} \\
0      & 0       & \cdots & w_m     & 0      & 0       & \cdots & v_n   

\end{array}\right) \begin{pmatrix}
a_1 \\
a_2\\
\vdots\\
a_m\\
b_1\\
b_2\\
\vdots \\
b_n
\end{pmatrix} = \begin{pmatrix}
z_1\\
z_2\\
\vdots
\\ 
z_m\\
z_{m+1} \\
z_{m+2} \\
\vdots \\
z_{m+n}

\end{pmatrix} \mod p,
\end{equation}
where for convenience we have set $v_0 = w_0=1$. The determinant of the matrix above, called the resultant, is known to equal \(\text{res}(v,w) = \displaystyle w_0^mv_0^n\prod_{i, j} \left(\nu_i-\mu_j \right)\), where $\nu_i, \mu_j$ are the roots of $v$ and $w$ over the algebraic closure of $\mathbb Z / p\mathbb Z$. In our starting factorization $u=\bar v\cdot \bar w $, the factors $\bar v, \bar w$ are relatively prime and so have no common roots and we have that $\text{res}(\bar v, \bar w)\neq 0$. At the same time we established that $v =\bar v \mod p$ and $w=\bar w\mod p $ and so $\text{res} (\bar v, \bar w) = \text{res}(v,w) \neq 0$. Since the determinant of \eqref{eqHanselLift2Sylvester} is non-zero, \eqref{eqHanselLift2Sylvester} has a unique solution. This shows both the existence and uniqueness of $v',w'$. This in turn concludes our inductive step.
\subsection{Hensel lift for more than two factors}
We want to extend the Hensel lift from the previous section to the case of more than two factors. Let
\[
u = \bar v_1 \dots \bar v_l \mod p
\]
be a factorization with \(p\)-prime. The considerations of the two factor case carry over to the multi-factor one. TODO(tmilev): spell out the notation. Equation \eqref{eqHenselLift0} becomes
\[
\begin{array}{rcl}
v_1'\dots v_k'& = & u + p^k \left(z+ a_1 v_2\dots v_l + v_1 a_2\dots v_l +\dots + v_1\dots v_{l-1} a_l   \right) +p^{2k} s
\end{array}
\]
for some polynomial $s$. Let $t_j$ be the polynomial obtained by removing the $j^{th}$ multiplicand from $v_1\dots v_l$, i.e., let
\[t_j=\frac{v_1\dots v_l}{v_j}. \]

Then equation \eqref{eqHanselLift2Sylvester} carries over directly. Its matrix becomes
\[
\left( \begin{array}{ccccccccccc}
t_{1,0}       & 0      & \cdots & 0              & t_{2,0}       & 0      & \cdots & 0              & \dots \\
\vdots        & \ddots & \ddots & \vdots         & \vdots        & \ddots & \ddots & \vdots         & \dots \\
\vdots        & \ddots & \ddots & 0              & \vdots        & \ddots & \ddots & 0              & \dots \\
t_{1,\deg t_1}& \ddots & \ddots & t_{1,0}        & t_{2,\deg t_2}& \ddots & \ddots & t_{2,0}        & \dots \\
0             & \ddots & \ddots & \vdots         & 0             & \ddots & \ddots & \vdots         & \dots \\
\vdots        & \ddots & \ddots & \vdots         & \vdots        & \ddots & \ddots & \vdots         & \dots \\
0             & \cdots & 0      & t_{1,\deg t_1} & 0             & \cdots & 0      & t_{2,\deg t_2} & \dots \\
\end{array}\right)\]

\end{document}

