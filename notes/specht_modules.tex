
(1) There exists a one to one correspondence between all permutations of (0..n-1) and all tableaux of size n, and, of course, to the lists of the integers (0..n-1).

(2) Permutations act on tableaux by shuffling the boxes.  
[[3 1], [2 5], 4] -- [(1,3,4)] -> [[2 1],[5 3], 4]

Permutations act on weighted sums of tableaux componentwise.


(3) The Young symmetrizer of the tableau T projects into the Specht module.  For the tableau [[1 2],[3 4], 5], it is
\[
\Sum_{CS=<S_{135},S_{24}} Sgn(\sigma)\sigma (\Sum_{RS=<S_{12},S_{34}} \tau v)
\]

(4) When we have a (?) vector v inside the Specht module, we act on it with each standard tableau.  This gives a basis for the Specht module.

(5) To get a matrix corresponding to a particular permutation p, we build the matrix [p(b0).. p(bₙ₋₁)] 

(6) Magically, these matrices look nice.

(7) Then the Bₙ case follows somehow.
