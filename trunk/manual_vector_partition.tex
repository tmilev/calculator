\documentclass{article}
\usepackage{amssymb}
\usepackage{hyperref}
\title{Working file for a ``Vector partition'' manual}
\author{Todor Milev}
\begin{document}
\maketitle
\tableofcontents

\section{Root subalgebra/Root subsystem tables}
In our table printouts, we use the following notation. 
\begin{itemize}
\item $g$ - the ambient simple complex Lie algebra.
\item $h$ - a fixed Cartan subalgebra over $\mathbb{C}$.
\item $k$ - a reductive subalgebra containing $h$. Thus $k$ is a reductive root subalgebra.
\item $k\_\{ss\}:=[k,k]$ - the semisimple (Levi) part of $k$
\item $C(k\_\{ss\}):=\{X\in g| [X,k]=0\}$ - the centralizer of $k\_ss$ in $g$. 
\item $A\_n, B\_n,C\_n, D\_n, E\_n, F\_n, G\_n$ - type of root system. The notation conventions follow \cite[Chapter III]{Humphreys}.
\item $A\_n'$ stands for a root subsystem consisting of short roots. The notation $A\_n$ is used only for a root subsystem consisting of long roots. 
\end{itemize}

\section{Vector partition functions}

\bibliographystyle{hamsalpha}
\bibliography{./TodorMilevsBibliography}


\end{document}
